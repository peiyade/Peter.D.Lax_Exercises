\documentclass[11pt]{elegantbook}

\title{Exercises of Peter D. Lax Functional analysis}
%\subtitle{Classic Elegant\LaTeX{} Template}

\author{Pei Yu}
%\institute{Elegant\LaTeX{} Program}
%\date{Aug 15, 2022}
%\version{4.4}
%\bioinfo{Bio}{Information}

%\extrainfo{Victory won\rq t come to us unless we go to it. }

\logo{logo-blue.png}
\cover{cover.jpg}

% modify the color in the middle of titlepage
\definecolor{customcolor}{RGB}{32,178,170}
\colorlet{coverlinecolor}{customcolor}
\usepackage{cprotect}

\addbibresource[location=local]{reference.bib} % bib




\begin{document}

\maketitle

\frontmatter
\tableofcontents

\mainmatter

\chapter{Linear spaces}


\chapter{Linear maps}

\section{Algebra of linear maps}

\section{Index of a linear map}

\chapter{The Hahn-Banach Theorem}

\section{The extension theorem}

\section{Geometric Hanh-Banach theorem}

Exercise 1.

\begin{proof}
  The if part. For any interior point $x$, there exists open ball $B(x)$ with radius $\epsilon>0$ and for some $y=\frac{1}{1+\frac{\epsilon}{2}}\in K$. Now we have
  \begin{equation*}
    p_K(x)\leqslant p_y(x)\leqslant \frac{1}{1+\frac{\epsilon}{2}} <1.
  \end{equation*}
  The only if part. For any $x$ satisfies $p_K(x)<1$, there exists interior point $y\in K$ s.t $p_y(x)<1$. Assume $B(y)$ with radius $\epsilon$ is an open subset in $K$, then for any ball $B(x)$ with radius $\epsilon\frac{|x|}{|y|}$ are in $K$ by convexity. 
\end{proof}

\section{Extensions of the Hanh-Banach theorem}

\chapter{Applications of the Hahn-Banach theorem}

\chapter{Normed linear spaces}

\chapter{Hilbert spaces}

\chapter{Applications of Hilbert space results}

\chapter{Duals of normed linear spaces}

\chapter{Applications of duality}

\chapter{Weak convergence}

\chapter{Applications of weak convergence}

\chapter{The weak and weak$^\ast$ topologies}

\chapter{Locally convex topologies and the Krein-Milman theorem}

\chapter{Examples of convex setsand their extreme points}

\chapter{Bounded linear maps}

\chapter{Examples of bounded linear maps}

\chapter{Banach algebras and their elementary spectral theorey}

\chapter{Gelfand's theory of commutative Banach algebras}

\chapter{Applications of Gelfand's theory of commutative Banach algebras}

\chapter{Examples of operators and their spectra}

\chapter{Compact maps}

\chapter{Examples of compact operators}

\chapter{Positive compact operators}

\chapter{Fredholm's theory of integral equations}

\chapter{Invariant subspaces}

\chapter{Harmonic analysis on a halfline}

\chapter{Index theory}

\chapter{Compact symmetric operators in Hilbert space}

\chapter{Examples of compact symmetric operators}

\chapter{Trace class and trace formula}

\chapter{Spectral theory of symmetric, normal, and unitary operators}

\chapter{Spectral theory of self-adjoint operators}

\chapter{Example of self-adjoint operators}

\chapter{Semigroups of operators}

\chapter{Groups of unitary operators}

\chapter{Examples of strongly continuous semigroups}

\chapter{Scattering theory}

\chapter{A theorem of beurling}
\end{document}
